%As we mentioned above, to achieve the best performance of the target model, the imitation parameter should be set to minimize the training error. 
In this section, we propose our method GDSDA-SVM that uses SVM as the base classifier and can effectively estimate the imitation parameter.
\subsection{Distillation with multiple sources}
As we have mentioned previously, imitation parameter is a hyperparameter in GDSDA. A common method to estimate the hyperparameter is to use cross-validation.
Here we show that it is possible to obtain a closed form cross-validation error\cite{cawley2006leave} in GDSDA-SVM.
As a result,  GDSDA-SVM can estimate the imitation parameter effectively with the gradient descent method.

In our GDSDA-SVM, instead of using hinge loss, we use Mean Squared Error (MSE) as our loss function to train the GDSDA-SVM for the following two reasons: (1) Many recently studies \cite{ba2014deep,luo2016face,romero2014fitnets,urban2016deep} show that MSE is an efficient measurement for the target model to distill the knowledge from the source model. (2) MSE can provide a closed form cross-validation error estimation, so we can estimate the imitation parameter more effectively. 

Suppose we have $L$ examples $\{\textbf{x}_j,\textbf{y}_j\}_{j=1}^L$ from $N$ classes in the target domain where $X\in R^{L\times d}, Y\in R^{L\times N}$. Meanwhile, there are $M-1$ the source (teacher) models providing the soft labels $Y^*=\{\textbf{y}^*_{ij}|j=1,...,L;i=1,...,M-1\}$ for each of the $L$ examples.
For simplicity, we concatenate the hard label $Y$ and soft label $Y^*$ into a new label matrix $S$ to denote them, where:
\[S=[Y;Y^*]=[Y_1;...;Y_M]; S \in R^{L\times M \times N}\]
To solve this $N$-class classification problem, we build $N$ binary SVMs.
To obtain the $n$th binary SVM, we have to solve the following optimization problem: 
\begin{equation}\label{eq:multi-distill}
\begin{aligned}
\underset{w_n}{\min} \qquad & \frac{1}{2}{|| \textbf{w}_n ||^2} + C\sum_{j}{e_{jn}^2} \quad
s.t. & e_{jn} = \sum_i\lambda_is_{ijn} - \textbf{w}_n\textbf{x}_j%;\sum_i\lambda_i=1;\\
%& \sum_i\lambda_i=1; \lambda_i \in [0,1]; i\in M;  j\in L\\
\end{aligned}  
\end{equation}
The Lagrangian of above optimization problem:
\begin{equation}
\begin{aligned}
\mathcal{L}=&\frac{1}{2}{|| \textbf{w}_n ||^2} + C\sum_{j} {e_{jn}^2}+\sum_{j}\eta_{jn}\left(\sum_i\lambda_is_{ijn} - \textbf{w}_n\textbf{x}_j-e_{jn}\right)%+\beta^{(n)}\left(\sum_i\lambda_i-1\right)
\end{aligned}
\end{equation}
To find the saddle point, 
\begin{equation}
\begin{aligned}
\frac{{\partial L}}{{\partial \textbf{w}_n}}& =0 \rightarrow \textbf{w}_n = \sum_{j}\eta_{jn} {\textbf{x}_j}; &
\frac{{\partial L}}{{\partial {e_{ijn}}}} & =0 \rightarrow \eta_{jn} = 2C\lambda_i {e_{ijn}}\\
\end{aligned}
\end{equation}
%For each example $\textbf{x}_j$ and its constraint of label $s_{ijn}$, we have $e_{ijn}  + \textbf{w}_n\textbf{x}_j= s_{ijn}$. Replacing $\textbf{w}_n$ and $e_{ijn}$,  we have:  
%\begin{equation}
%\begin{aligned}
%%x_j\sum_{k}\alpha^{(n)}_{ik}x_k+\frac{\alpha^{(n)}_{ij}}{2C\lambda_i}&=s_{ijn}\\
%\lambda_i\textbf{x}_j\sum_{k}\alpha^{(n)}_{ik}\textbf{x}_k+\frac{\alpha^{(n)}_{ij}}{2C}&=\lambda_is_{ijn}
%%x_j\sum_{k}\alpha^{(n)}_{ik}x_k+\sum_j\frac{\alpha^{(n)}_{ij}}{2C}&=\sum_j\lambda_js_{ijn}
%\end{aligned}
%\end{equation}
For each example $\textbf{x}_j$ and its constraint of label $s_{ijn}$, we have $e_{jn}  + \textbf{w}_n\textbf{x}_j= \sum_i\lambda_is_{ijn}$. Replacing $\textbf{w}_n$ and $e_{ijn}$,  we have:
%\begin{equation}
%\underbrace{\sum_i\lambda_i}_{=1} \textbf{x}_j\sum_{k}\alpha^{(n)}_{ik}\textbf{x}_k+\sum_i\frac{\alpha^{(n)}_{ij}}{2C}=\sum_i\lambda_is_{ijn}
%\end{equation}
%%Here we use $K$ to denote the kernel matrix $K=\{x_ix_j|i,j\in 1\dots L\}$.
%Let %$M=[K+\frac{\mathbf{I}}{2C}]$ and 
%$\eta_{jn}=\sum_i\alpha^{(n)}_{ij}$, we have:
\begin{equation}
\begin{aligned}
\textbf{x}_j\sum_k\eta_{kn}\textbf{x}_k+\frac{\eta_{jn}}{2C}&=\sum_i\lambda_is_{ijn}\\
%M\eta_n&=S_n\begin{bmatrix}
%\lambda_1\\\vdots\\\lambda_m
%\end{bmatrix}
\end{aligned}
\end{equation}

Here we use $\Omega$ to denote the matrix $\Omega=[K+\frac{\mathbf{I}}{2C}]$ where $K$ is the linear kernel matrix $K=\{\textbf{x}_i\textbf{x}_j|i,j\in 1\dots L\}$. To simplify our notation, let ${\eta}'_{n}=M^{-1}S_n$ where $S_n$ is the matrix $S_n=\{s_{ijn}|i\in M;j\in L\}$ and $\Omega^{-1}$ is the inverse of matrix $\Omega$. 

According to  \cite{cawley2006leave}, the Leave-one-out estimation of the example $\textbf{x}_j$ for the $n$th binary SVM can be written as:
%\[\sum_i\lambda_is_{ijn}-\hat{y}_{jn} =\frac{{\eta}_{jn}}{\Omega_{jj}^{-1}} =\frac{\sum_i\lambda_i{\eta}'_{ijn}}{\Omega_{jj}^{-1}}\]
\begin{equation}\label{eq:yhat}
\hat{y}_{jn} = \sum_i\lambda_is_{ijn}-\frac{{\eta}_{jn}}{\Omega_{jj}^{-1}}
\end{equation}
where $\Omega^{-1}_{jj}$ is the $j$th diagonal element of $\Omega^{-1}$.
% Now for any given $\lambda$, we have found an efficient way to estimate the LOO of each binary target model for example $\textbf{x}_j$. 
%In the following part, we will introduce how to find the optimal $\lambda_i$ for each of the source models. 
\subsection{Cross-entropy loss for imitation parameter estimation}
From the previous part, we have already found a effective way to calculate the leave-one-out estimation of the target model. The optimal imitation parameters can be found by minimizing the leave-one-out cross-validation error on the target data:
\begin{equation}\label{eq:loo_loss}
\begin{aligned}
\min \quad& L_c\left(\lambda\right)=\frac{1}{2}\sum_i^M||\lambda_i||^2+\frac{1}{L}\sum_{j,n}\ell\left(y_{in},\hat{y}_{jn}\left(\lambda\right)\right) &
%s.t. \quad& \sum\lambda_i=1
\end{aligned}
\end{equation}
Here we use the $\ell$-2 regularization term to control the complexity of $\lambda$ so that the target model can achieve better generalization performance even with a small training set. For the loss function $\ell(\cdot,\cdot)$, we use the cross-entropy loss function.
\begin{equation}\label{eq:ce}
\begin{aligned}
\ell\left(y_{in},\hat{y}_{jn}\left(\lambda\right)\right)=&y_{in}\log(P_{jn}) & for \quad&
P_{jn} = \frac{e^{\hat{y}_{jn}}}{\sum_{h} e^{\hat{y}_{jh}}}
\end{aligned}
\end{equation}
Typically, cross-entropy pays less attention to a single incorrect prediction which reduces the affect of the outliers of the training data. Moreover, cross-entropy has its own advantage with our "fake label" strategy. As we have mentioned previously, we use gray code to encode the unlabeled examples. When we use cross-entropy loss, it can automatically ignore penalties of the unlabeled examples and reduce the affect of the noise introduced by our "fake label" strategy. 
%Let:
%\begin{equation}\label{eq:mu}
%\begin{aligned}
%\mu_{ijn}=s_{ijn}-\frac{{\eta}'_{ijn}}{\Omega_{jj}^{-1}} \qquad
%\end{aligned}
%\end{equation}
As a result, the derivative of Eq. \eqref{eq:ce} can be calculated as:
\begin{equation}\label{eq:p}
\begin{aligned}
\frac{\partial \ell(\lambda)}{\partial \lambda}&=\sum_{j,n}s_{ijn}\left(P_{jn}-{y}_{jn}\right)
\end{aligned}
\end{equation}
\begin{algorithm}[t]
	\caption{GDSDA-SVM}\label{alg:svm}
	\begin{algorithmic}
		\REQUIRE Input examples $X=\{\textbf{x}_1,...,\textbf{x}_L\}$, number of classes $N$, number of sources $M$, 3-D label matrix, $S=[Y_1,Y_2,...,Y_{M}]$ with size $L\times M \times N$, temperature $T$ %optimization iteration $iter$
		\ENSURE Target model $f_t = Wx$
		\STATE Compute $\Omega=[K+\frac{\mathbf{I}}{2C}]$
		\STATE Compute imitation parameter $\lambda$ with Algorithm \ref{alg:lambda}
		\STATE Generate the new label $Y_{new}=\sum_i\lambda_iY_i$
		\STATE Compute $\eta = \Omega^{-1}Y_{new}$
		\STATE Compute $w_n = \sum_j \eta_{jn}x_j$
	\end{algorithmic}	
\end{algorithm}
\begin{algorithm}[t]
	\caption{$\lambda$ Optimization}\label{alg:lambda}
\begin{algorithmic}
	\REQUIRE Input examples $X$, number of classes $N$, size of sources $M$, 3D label matrix $S$,  optimization iteration $iter$, Kernel matrix $\Omega$
    \ENSURE Imitation parameter $\lambda$
    \STATE Initialize $\lambda = \frac{1}{M}$, 
    
    \STATE Let $S_n$ be the label matrix of $S$ for class $n$
    \FOR{Each label $S_n$} 
    \STATE Compute $\eta'_n=\Omega^{-1}S_n$ 
    \ENDFOR
%    \STATE Compute $\mu_{ijn}=s_{ijn}-\frac{{\eta}'_{ijn}}{\Omega_{jj}^{-1}}$
    \FOR {$it \in \{1,...,iter\}$ }
	    \STATE Compute $\hat{y}_{jn}$ and $P_{jn}$ with \eqref{eq:yhat}  and \eqref{eq:ce}
%	    \STATE $\Delta_{\lambda} \leftarrow 0$
	    \FOR {each $\textbf{x}_j$ in $X$}
		    \STATE $\Delta_{\lambda} = \Delta_{\lambda}+\sum_{j,n}s_{ijn}\left(P_{jn}-{y}_{jn}\right)$
	    \ENDFOR
	    \STATE $\Delta_{\lambda} =\Delta_{\lambda}/L$, $\lambda = \lambda - \frac{1}{it}(\Delta_{\lambda}+\lambda)$
	    % $\lambda = \lambda / \sum\lambda_i$
%	    \STATE 
    \ENDFOR
\end{algorithmic}	
\end{algorithm}

To summarize, we describe GDSDA-SVM in Algorithm \ref{alg:svm}. As the optimization problem \eqref{eq:loo_loss} is strongly convex, we can prove that Algorithm \ref{alg:lambda} can converge to the optimal $\lambda$ with the rate of $O(\log(t)/t)$ where $t$ is the optimization iteration (We are not able to show our proof here due to the space limit). 




